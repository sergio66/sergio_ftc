
% RTP Format Specification & User's Guide
%
% H. Motteler
%

\documentclass[12pt]{article}

\usepackage{graphicx}

\DeclareGraphicsExtensions{.pdf}  %for .pdf version
%\DeclareGraphicsExtensions{.eps} %for .ps version

\newif\iftth

\iftth
\begin{html}
<title>RTP Format Specification</title>
<BODY BGCOLOR="#FFFFFF">
\end{html}
\fi

\title{{\bf RTP Format Specification\\
		and\\
	User's Guide}\bigskip\bigskip \\
        {Version 1.05}\bigskip \\ }

\author{Howard~E.~Motteler}

\date{\today\vspace{1cm}}

\begin{document}

\maketitle

\begin{abstract}

We present a data format for driving radiative transfer calculations
and manipulating atmospheric profiles.  Calculated and observed
radiances may be included as optional fields, allowing for the
representation of basic co-location datasets.  An implementation as
HDF~4 Vdatas is given, including Fortran, C, and Matlab application
interfaces.

\end{abstract}

% \newpage

%---------------------
\section{Introduction}
%---------------------

The ``Radiative Transfer Profile'' (RTP) format is a data format for
sets of atmospheric profiles, optionally paired with calculated
and/or observed radiances.  The format consists of a header record
and an array of profile records.  It was derived from the GENLN2
user profile format, extened with selected AIRS level~2 field
definitions.  RTP is currently implemented as HDF~4 Vdatas and as
structure arrays in Fortran, C, and Matlab.

The format is intended to give a well-defined interface to radiative
transfer codes, allowing for the specification of just the
information needed for such calculations.  It allow for modularity
of both radiative transfer codes and of other tools for manipulating
profiles, including tools for field selection, level interpolation
and level-to-layer translations, translation of units, and building
composite profiles from multiple sources.  The RTP specification has
some flexibility in the field set actually saved to disk, both to
save space and to provide compatibility across file versions.  The
optional observation fields may be used to build simple co-location
datasets.


%--------------------------------------
\section{The RTP format definition}
%--------------------------------------

The RTP format consists of a header record with information about
all the profiles in a file, and one or more profiles saved as an
array of records.  Field definitions for the header and profile
records are given below.  These names are both the names of the
Vdata fields and the Fortran, C, and Matlab structure fields, with
the exception of the constituent arrays, as discussed below.
Depending on the application, only a subset of the fields described
here need be present in an RTP file.  Fields are matched by field
name, and no particular order for the header or profile fields is
assumed.


%%%%%%%%%%%%%%%%%%%%%%%%%%%%%%%%%%%%%%%%%%%%%%%%%%%%%%%%%%%%%%%%%%%%%%%%%%
\begin{figure}
\vskip-14mm
{\small
\begin{verbatim}

                      RTP Header Fields

 field name   short description       data type         units
 ----------   ------------------     -----------       -------
 ptype        profile type           scalar int32     see note [1]
 pfields      profile field set      scalar int32     see note [2]

 pmin         min plevs value        scalar float32   millibars
 pmax         max plevs value        scalar float32   millibars
 ngas         number of gases        scalar int32     [0,MAXGAS]
 glist        constituent gas list   ngas   int32     HITRAN gas ID
 gunit        constituent gas units  ngas   int32     gas unit code

 nchan        number of channels     scalar int32     count
 ichan        channel numbers        nchan  int32     [0,MAXCHAN]
 vchan        channel center freq.   nchan  float32   1/cm
 vcmin        channel set min freq.  scalar float32   1/cm
 vcmax        channel set max freq.  scalar float32   1/cm

 mwnchan      number of MW channels  scalar int32     count
 mwfchan      MW channel freq's      mwnchan float32  GHZ

 udef         user-defined array     MAXUDEF float32  undefined
 udef1        user-defined field     scalar float32   undefined
 udef2        user-defined field     scalar float32   undefined


Notes:

 [1] ptype values are
        1. level profile       LEVPRO = 0
        2. layer profile       LAYPRO = 1
        3. AIRS pseudo-layers  AIRSLAY = 2

 [2] RTP profile fields are organized in five groups
        1. profile data              PROFBIT = 1
        2. calculated IR radiances   IRCALCBIT = 2
        3. observed IR radiances     IROBSVBIT = 4
        4. calculated MW Tb          MWCALCBIT = 8
        5. observed MW Tb            MWOBSVBIT = 16
     For example, a profile with both calculated and observed IR
     radiances would have pfields = PROFBIT + IRCALCBIT + IROBSVBIT

\end{verbatim}
}
\end{figure}
%%%%%%%%%%%%%%%%%%%%%%%%%%%%%%%%%%%%%%%%%%%%%%%%%%%%%%%%%%%%%%%%%%%%%%%%%%
\begin{figure}
{\small
\begin{verbatim}

                Profile Fields -- Surface Data
  
 field name   short description       data type         units
 ----------   ------------------     -----------       -------
 plat         profile latitude       scalar float32   [-90,90] deg.
 plon         profile longitude      scalar float32   [-180,360] deg.
 ptime        profile time           scalar float64   TAI  

 stemp        surface temperature    scalar float32   Kelvins
 nrho         number of refl. pts    scalar int32     [0,MAXRHO]
 rho          surface reflectance    nrho   float32   [0,1]
 rfreq        reflectance freq's     nrho   float32   1/cm
 nemis        number of emis. pts    scalar int32     [0,MAXEMIS]
 emis         surface emissivities   nemis  float32   [0,1]
 efreq        emissivity freq's      nemis  float32   1/cm

 salti        surface altitude       scalar float32   meters
 spres        surface pressure       scalar float32   millibars
 smoist       soil moisture fract.   scalar float32   [0,1]
 landfrac     land fraction          scalar float32   [0,1]
 landtype     land type code         scalar int32     see text

\end{verbatim}
}
\end{figure}
%%%%%%%%%%%%%%%%%%%%%%%%%%%%%%%%%%%%%%%%%%%%%%%%%%%%%%%%%%%%%%%%%%%%%%%%%%
\begin{figure}
{\small
\begin{verbatim}

                Profile Fields -- MW Surface Data

 field name   short description       data type         units
 ----------   ------------------     -----------       -------
 mwnemis      number of MW emis pts  scalar int32     [0,MWMAXEMIS]
 mwefreq      MW emissivity freq's   mwnemis float32  GHZ
 mwemis       MW emissivities        mwnemis float32  [0,1]
 mwnstb       number of MW surf pts  scalar int32     [0,MWMAXSTB]
 mwsfreq      MW surface Tb freq's   mwnstb float32   GHZ
 mwstb        MW surface Tbs         mwnstb float32   kelvins

\end{verbatim}
}
\end{figure}
%%%%%%%%%%%%%%%%%%%%%%%%%%%%%%%%%%%%%%%%%%%%%%%%%%%%%%%%%%%%%%%%%%%%%%%%%%
\begin{figure}
{\small
\begin{verbatim}

                Profile Fields -- Atmospheric Data

 field name   short description       data type         units
 ----------   ------------------     -----------       -------
 nlevs        number of press lev's  scalar int32     [0,MAXLEV]
 plevs        pressure levels        nlevs  float32   millibars
 plays        pressure layers        nlevs  float32   millibars
 palts        level altitudes        nlevs  float32   meters
 ptemp        temperature profile    nlevs  float32   Kelvins
 gas_<i> [1]  constituent amounts    nlevs  float32   PPMV
 gxover       gas crossover press    ngas   float32   millibars
 txover       temp crossover press   scalar float32   millibars
 co2ppm       CO2 mixing ratio       scalar float32   PPMV

 cfrac        cloud fraction         scalar float32   [0,1]
 ctype        cloud type code        scalar int32     see note [2]
 cemis        cloud top emissivity   scalar float32   [0,1]
 cprtop       cloud top pressure     scalar float32   millibars
 cprbot       cloud bottom pressure  scalar float32   millibars
 cngwat       cloud non-gas water    scalar float32   PPMV
 cpsize       cloud particle size    scalar float32   microns

 wspeed       wind speed             scalar float32   meters/sec
 wsource      wind source            scalar float32   [-180,360] deg.


Notes:

 [1] There is one field here for each constituent in a file; the 
     constituents are listed in the header field glist.  The Fortran 
     API presents this data as an ngas x nlevs array.

 [2] Cloud type codes are: 1 = cirrus, 2 = [fill out this list]

\end{verbatim}
}
\end{figure}
%%%%%%%%%%%%%%%%%%%%%%%%%%%%%%%%%%%%%%%%%%%%%%%%%%%%%%%%%%%%%%%%%%%%%%%%%%
\begin{figure}
{\small
\begin{verbatim}

              Profile Fields -- Common Radiance Data
  
 field name   short description       data type         units
 ----------   ------------------     -----------       -------
 pobs         observer pressure      scalar float32   millibars
 zobs         observer height        scalar float32   meters
 upwell       radiation direction    scalar int32     1=up, 2=down

 scanang      IR scan/view angle     scalar float32   [-90,90] deg.
 satzen       IR zenith angle        scalar float32   [0,90] deg.
 satazi       IR azimuth angle       scalar float32   [-180,360] deg.
 solzen       sun zenith angle       scalar float32   [0,90] deg.
 solazi       sun azimuth angle      scalar float32   [-180,360] deg.

 mwasang      AMSU-A scan/view ang   scalar float32   [-90,90] deg.
 mwaszen      AMSU-A zenith angle    scalar float32   [0,90] deg.
 mwbsang      AMSU-B scan/view ang   scalar float32   [-90,90] deg.
 mwbszen      AMSU-B zenith angle    scalar float32   [0,90] deg.

\end{verbatim}
}
\end{figure}
%%%%%%%%%%%%%%%%%%%%%%%%%%%%%%%%%%%%%%%%%%%%%%%%%%%%%%%%%%%%%%%%%%%%%%%%%%
\begin{figure}
{\small
\begin{verbatim}

              Profile Fields -- Calculated Radiance Data
  
 field name   short description       data type         units
 ----------   ------------------     -----------       -------
 rcalc        calculated IR rad.     nchan  float32   mW/m^2/cm^-1/str
 mwcalc       calculated MW BT       mwnchan float32  kelvins

\end{verbatim}
}
\end{figure}
%%%%%%%%%%%%%%%%%%%%%%%%%%%%%%%%%%%%%%%%%%%%%%%%%%%%%%%%%%%%%%%%%%%%%%%%%%
\begin{figure}
{\small
\begin{verbatim}

              Profile Fields -- Observed Radiance Data
  
 field name   short description       data type         units
 ----------   ------------------     -----------       -------
 rlat         radiance obs lat.      scalar float32   [-90,90] deg.
 rlon         radiance obs lon.      scalar float32   [-180,360] deg. 
 rtime        radiance obs time      scalar float64   TAI
 robs1        observed IR rad.       nchan  float32   mW/m^2/cm^-1/str
 calflag      calibration flag       nchan  uchar8    see text
 irinst       IR instrument code     scalar int32     instrument code

 mwobs        observed MW BT         mwnchan float32  kelvins
 mwinst       MW instrument code     scalar int32     instrument code
 findex       file (granule) index   scalar int32     index
 atrack       along-track index      scalar int32     index
 xtrack       cross-track index      scalar int32     index

\end{verbatim}
}
\end{figure}
%%%%%%%%%%%%%%%%%%%%%%%%%%%%%%%%%%%%%%%%%%%%%%%%%%%%%%%%%%%%%%%%%%%%%%%%%%
\begin{figure}
{\small
\begin{verbatim}

              Profile Fields -- User Defined Data
  
 field name   short description       data type         units
 ----------   ------------------     -----------       -------
 pnote        profile annotation     MAXPNOTE char8   text
 udef         user-defined array     MAXUDEF float32  undefined
 udef1        spare, user-defined    scalar float32   undefined
 udef2        spare, user-defined    scalar float32   undefined

\end{verbatim}
}
\end{figure}
%%%%%%%%%%%%%%%%%%%%%%%%%%%%%%%%%%%%%%%%%%%%%%%%%%%%%%%%%%%%%%%%%%%%%%%%%%

\subsection{Levels and Layers}

For level profiles, {\tt nlevs} is the number of levels and {\tt
plevs} the pressure levels; the {\tt nlevs} temperature and
constituent fields contain level values.  For layer profiles, {\tt
nlevs} is the number of layer boundaries, {\tt plevs} is the
boundary pressures, and the {\tt nlevs}$ - 1$ temperature and
constituent fields contain layer values.  The optional {\tt plays}
field may be used for a nominal layer pressure, if desired.  The
{\tt palts} field, if used, is altitudes for the pressure levels,
for either level or layer profiles.

The header field {\tt ptype} flags the profile as being a level
profile, a layer profile, or a profile using AIRS pseudo-layers,
with the following values.

{\small
\begin{verbatim}
    1. level profile        LEVPRO = 0
    2. layer profile        LAYPRO = 1
    3. AIRS pseudo-layers   AIRSLAY = 2
\end{verbatim}
}

A convention that lower indices correspond to lower pressures is
suggested but not required.  The header fields {\tt pmax} and {\tt
pmin} are intended to hold the max and min level pressures over all
profiles, or some upper and lower bound on these values.

\subsection{Constituents}

Constituent fields are named with their HITRAN gas ID's, with {\tt
gas\_1} water, {\tt gas\_2} CO$_2$, and so on.  A list of HITRAN
gas ID's is given in an appendix.  The header field {\tt glist}
gives a list of the constituent ID's for the constituents present
in the file.  The default constituent unit is PPMV.

The Fortran and C application interfaces represent constituents as
a 2D array {\tt gamnt} whose rows are layers and whose columns are
gas ID index, rather than as a set of separate fields {\tt
gas\_<i>} as they are actually saved in the file; the {\tt
gas\_<i>} fields are the columns of the 2D {\tt gamnt} array.

There are a wide variety of constituent units in current use; in
consideration of this we have added a {\tt gunit} array to the
header, assigning a unit code for each constituent and allowing at
least the potential for automatic conversions.  These unit codes are
given in {\tt gas\_units\_code.txt}.  This file is included in the
RTP distribution, but the most up-to-date version can be found in
the SARTA fast radiative transfer package.

Note that only a small subset of possible constituents are typically
recognized and processed by fast models for radiative transfer
calculation, typically water, ozone, and perhaps methane, CO$_2$,
and CO; see the documentation of the relevant radiative transfer
code for more information.


\subsection{Field Sets and Sizes}

Individual profiles may have varying pressures levels, emissivity,
reflectance, and surface brightness temperature sets.  All profiles
in a file are assumed to have the same constituent set, and if
radiances are present all profiles have the same channel set.

RTP fields may be scalars or one-dimensional arrays; this is a
limitation of the underlying HDF Vdata format.  Most arrays have an
associated size field.  If this size field is in the header, as in
the case of {\tt ngas} or {\tt nchan} then it is assumed to be the
same for all profiles, while if the size field is in a profile, as
in the case of {\tt nlevs} or {\tt nemis}, then it applies only to
that profile.

The size of array fields in the RTP HDF Vdata implementation may in
some cases be bigger than what is specified by the associated size
field.  This can happen because the HDF Vdata format requires a
single size be associated with each field, which then has to be at
least the max of all the actual field sizes.  Because of this, when
a size-field is avaliable its value should be used instead of the
possibly larger Vdata field size.

The field set for RTP is not required to be fixed to precisely the
fields listed here.  Fields are matched by field name, and no
particular order for the header or profile fields is assumed


\subsection{Field Groups}

The {\tt pfields} field in the header is used by the C/Fortran API
to control what which field groups will be written to a file.  Profile
fields are organized as five groups,

\begin{verbatim}
   1. profile data                      PROFBIT = 1
   2. calculated IR radiances           IRCALCBIT = 2
   3. observed IR radiances             IROBSVBIT = 4
   4. calculated MW brightness temp.    MWCALCBIT = 8
   5. observed MW brightness temp.      MWOBSVBIT = 16
\end{verbatim}

These groups can occur in any combination.  The associated numbers
are bit fields, set in pfields if the associated data is present
in the file.  Thus for example profile data with calculated and
observed IR radiances would be represented as
{\tt pfields}~$=$ {\tt PROFBIT}~$+$ {\tt IRCALCBIT}~$+$ {\tt
IROBSVBIT},
while a profile with calculated and observed MW radiances but no
IR data would have 
{\tt pfields}~$=$ {\tt PROFBIT}~$+$ {\tt MWCALCBIT}~$+$ {\tt
MWOBSVBIT}.

Note that we can have {\tt nchan}~$>0$ and channel data in the
header without having either calculated or observed radiances in a
file, to specify a set of channels whose radiances are to be
calculated later.


\subsection{HDF Attributes}

Attributes are associated either with the header or with the
profile record set, and have three parts: the field the attribute
is associated with, the attribute name, and the attribute text.  In
addition to proper field names, the field name ``header'' is used
for general header attributes, and ``profile'' for general profile
attributes.

RTP attributes should typically include such information as title,
author, date, and at least a brief descriptive comment.  This
general information should be set as attributes of the header
record.  Note that the Fortran/C API uses the 2D {\tt gamnt} array
for constituents; this is not actually a Vdata field, and so can not
take an attribute.  Attributes may be attached to individual
constituents with their {\tt gas\_<i>} names, where {\tt <i>} is the
HITRAN gas ID.


\section{Application Interfaces}


\subsection{The Fortran and C API}

The Fortran API consists of four routines: {\tt rtpopen}, {\tt
rtpread}, {\tt rtpwrite}, and {\tt rtpclose}.  Documentation for
these is included in an appendix.  The Fortran API uses static
structures whose fields, with a few exceptions noted below, are the
same as the RTP fields defined above.  Normally, only a subset of
the Fortran structure fields will be written, with the header field
{\tt pfields} and the header size fields used to determine what
actually goes into a file.  When reading data, if a file contains
header or profile fields not in the Fortran structure definition,
they are simply ignored.  Fields that are defined in the Fortran
structure but are not in a file are returned as ``BAD'', or with the
first element BAD, for vectors, while missing size fields are
returned as zero.

Attributes are passed to and from the Fortran API in the {\tt
RTPATTR} structure array.  The records in this array have three
fields: {\tt fname}, the field name the attribute is to be
associated with, {\tt aname}, the attribute name, and {\tt atext},
the attribute text.  The header attribute field name should be
either ``header'', for a general attribute or comment, or a
particular header field name.  Similarly, the attribute profile
field name should be either ``profiles'' or a specific profile
field.  Attribute strings need to be null-terminated, with char(0),
and the record after the last valid record in an attribute set
should have fname set to char(0).  See ftest1.f for and ftest2.f
examples of reading, writing, and updating attributes.

The Fortran structures differer from the Vdata fields in two ways.
First, instead of a {\tt gas\_<i>} profile field for each
constituent, the Fortran API uses a single array {\tt
gamnt(MAXLEV,MAXGAS)} to pass constituent amounts; the {\tt
gas\_<i>} fields from the HDF file are the columns of this array.
Second, the Fortran/C RTP header structure includes the following
max size fields, which are not actually written to the Vdata header.

\begin{verbatim}
 mlevs     max number of levels   scalar int32   [0,MAXLEV]
 mrho      max num of refl pts    scalar int32   [0,MAXRHO]
 memis     max num of emis pts    scalar int32   [0,MAXEMIS]
 mwmemis   max MW emis pts        scalar int32   [0,MWMAXEMIS]
 mwmstb    max MW sTb pts         scalar int32   [0,MWMAXSTB]
\end{verbatim}

On a read, these fields are set to the associated profile Vdata
field sizes.  On a write, they are used to to set the size of the
associated Vdata profile fields.  They can simply be set to the MAX
limits, or to zero if the fields are not used; but using an actual
max for the profile set, particularly for mlevs, can give a
significant space savings.

A makefile is supplied to build the RTP API routines as a library
file librtp.a.  A Fortran demo makefile, ``Makefile.f77'' is also
provided, to compile the F77 demo programs ftest1.f and ftest2.f
and link them with the RTP libraries.


\subsection{The Matlab API}

The RTP Matlab implementation is a fairly direct mapping between
Matlab structure arrays and HDF~4 Vdatas.  A read will only return
those fields that are in the HDF Vdata, and a write will only write
the fields in the Matlab structure.  The Matlab RTP API is available
as part of the ASL package ``h4tools''; see the README file there
for more information.


\subsection{Data Types}

Most RPT fields are either 32-bit integers or 32-bit floats, as
noted in the field tables, with the exception of the time fields
which are 64-bit floats, and the pnote and calflag fields, which are
character arrays.  The HDF C types are defined in the HDF include
file ``hdf.h''.

{\small
\begin{verbatim}
   HDF type codes     HDF C types      Fortran types
    DFNT_INT32          int32           integer*4
    DFNT_FLOAT32        float32         real*4
    DFNT_FLOAT64        float64         real*8
    DFNT_CHAR8          char8           character*<n>
    DFNT_UCHAR8         uchar8          character*<n>
\end{verbatim}
}

%%%%%%%%%%%%%%%%%%%%%%%%%%%%%%%%%%%%%%%%%%%%%%%%%%%%%%%%%%%%%%%%%%%%%%%%%%
\begin{figure}
{\small
\begin{verbatim}

NAME

  rtpopen -- Fortran interface to open RTP files

SUMMARY

  rtpopen() is used to open an HDF RTP ("Radiative Transfer Profile")
  file for reading or writing profile data.  In addition, it reads or
  writes RTP header data and HDF header and profile attributes. 

FORTRAN PARAMETERS

  data type             name      short description       direction
  ---------             -----     -----------------       ---------
  CHARACTER *(*)        fname     RTP file name              IN
  CHARACTER *(*)        mode      'c'=create, 'r'=read       IN
  STRUCTURE /RTPHEAD/   head      RTP header structure       IN/OUT
  STRUCTURE /RTPATTR/   hfatt     RTP header attributes      IN/OUT
  STRUCTURE /RTPATTR/   pfatt     RTP profile attributes     IN/OUT
  INTEGER               rchan     RTP profile channel        OUT

VALUE RETURNED

  0 if successful, -1 on errors

INCLUDE FILES

  rtpdefs.f -- Fortran header, profile, and attribute structures

DISCUSSION

  The valid open modes are 'r' to read an existing file and 'c' to
  create a new file.

  HDF attributes are read and written in an array of RTPATTR
  structures, with one structure record per attribute.  Attributes
  should be terminated with char(0), and are returned that way, for
  a read.  The end of the attribute array is flagged with a char(0)
  at the beginning of the fname field.

\end{verbatim}
}
\end{figure}
%%%%%%%%%%%%%%%%%%%%%%%%%%%%%%%%%%%%%%%%%%%%%%%%%%%%%%%%%%%%%%%%%%%%%%%%%%
\begin{figure}
{\small
\begin{verbatim}

NAME    

  rtpread -- Fortran interface to read an RTP profile

SUMMARY

  rtpread reads a profile from an open RTP channel, and returns
  the data in the RTPPROF structure.  Successive calls to rtpread
  return successive profiles from the file, with -1 returned on
  EOF.


FORTRAN PARAMETERS

  data type             name      short description       direction
  ---------             -----     -----------------       ---------
  INTEGER               rchan     RTP profile channel         IN
  STRUCTURE /RTPPROF/   prof      RTP profile structure       OUT

VALUE RETURNED

  1 (the number of profiles read) on success , -1 on errors or EOF

\end{verbatim}
}
\end{figure}
%%%%%%%%%%%%%%%%%%%%%%%%%%%%%%%%%%%%%%%%%%%%%%%%%%%%%%%%%%%%%%%%%%%%%%%%%%
\begin{figure}
{\small
\begin{verbatim}

NAME    

  rtpwrite -- Fortran interface to write an RTP profile

SUMMARY

  rtpwrite writes an RTP profile, represented as the contents
  of an RTPPROF structure, to an open RTP channel.  Successive
  calls write successive profiles.

FORTRAN PARAMETERS

  data type             name      short description       direction
  ---------             -----     -----------------       ---------
  INTEGER               rchan     RTP profile channel         IN
  STRUCTURE /RTPPROF/   prof      RTP profile structure       IN

VALUE RETURNED

  0 on success, -1 on errors

\end{verbatim}
}
\end{figure}
%%%%%%%%%%%%%%%%%%%%%%%%%%%%%%%%%%%%%%%%%%%%%%%%%%%%%%%%%%%%%%%%%%%%%%%%%%
\begin{figure}
{\small
\begin{verbatim}

NAME    

  rtpclose -- Fortran interface to close an RTP open channel

SUMMARY

  rtpclose finishes up after reading or writing an RTP file, 
  writing out any buffers and closing the HDF interface

FORTRAN PARAMETERS

  data type     name      short description      direction
  ---------     -----     -----------------      ---------
  INTEGER       rchan     RTP profile channel       IN

VALUE RETURNED

  0 on success, -1 on errors

\end{verbatim}
}
\end{figure}
%%%%%%%%%%%%%%%%%%%%%%%%%%%%%%%%%%%%%%%%%%%%%%%%%%%%%%%%%%%%%%%%%%%%%%%%%%
\begin{figure}
{\small
\begin{verbatim}

NAME    

  rtpinit -- initialze RTP profile and header structures

SUMMARY

  rtpinit initializes RTP profile structures with some sensible
  default vaules, and is used when creating a new profile set; it
  should generally not be used when modifying existing profiles.

  rtpinit sets all field sizes to zero, and all data values to
  "BAD", so that only actual values and sizes need to be written

FORTRAN PARAMETERS

  data type             name      short description       direction
  ---------             -----     -----------------       ---------
  STRUCTURE /RTPHEAD/   head      RTP header structure       OUT
  STRUCTURE /RTPPROF/   prof      RTP profile structure      OUT

VALUE RETURNED

  rtpinit always returns 0

\end{verbatim}
}
\end{figure}
%%%%%%%%%%%%%%%%%%%%%%%%%%%%%%%%%%%%%%%%%%%%%%%%%%%%%%%%%%%%%%%%%%%%%%%%%%
\begin{figure}
{\small
\begin{verbatim}

   NAME

     rtpdump -- basic RTP dump utility
  
   USAGE
  
     rtpdump [-achp] [-n k] rtpfile
  
   OPTIONS
  
     -a      dump attributes
     -c      dump RTP channel info
     -h      dump header structure
     -p      dump profile structure
     -n <k>  select profile <k> for channel or profile 
             structure dumps; the first profile is 1
  
   BUGS

     the output is from debug and error dump routines and is not very
     fancy; the -p option only prints a subset of profile fields
 
\end{verbatim}
}
\end{figure}
%%%%%%%%%%%%%%%%%%%%%%%%%%%%%%%%%%%%%%%%%%%%%%%%%%%%%%%%%%%%%%%%%%%%%%%%%%  
\begin{figure}
{\small
\begin{verbatim}

                   HITRAN Gas List
                   ---------------
                                    
    Gases from the 1998 HITRAN line database
    
     1 = H2O (water vapor)          17 = HI
     2 = CO2                        18 = ClO
     3 = O3 (ozone)                 19 = OCS
     4 = N2O                        20 = H2CO
     5 = CO                         21 = HOCl
     6 = CH4 (methane)              22 = N2 (nitrogen)
     7 = O2 (oxygen)                23 = HCN
     8 = NO                         24 = CH3Cl
     9 = SO2                        25 = H2O2
    10 = NO2                        26 = C2H2
    11 = NH3 (ammonia)              27 = C2H6
    12 = HNO3                       28 = PH3
    13 = OH                         29 = COF2
    14 = HF                         30 = SF6
    15 = HCl                        31 = H2S
    16 = HBr
    
    Gases represented only by cross-sections
    
    51 = CCl3F (CFC-11)             58 = C2Cl2F4 (CFC-114)
    52 = CCl2F2 (CFC-12)            59 = C2ClF5 (CFC-115)
    53 = CClF3 (CFC-13)             60 = CCl4
    54 = CF4 (CFC-14)               61 = ClONO2
    55 = CHCl2F (CFC-21)            62 = N2O5
    56 = CHClF2 (CFC-22)            63 = HNO4
    57 = C2Cl3F3 (CFC-113)

  
\end{verbatim}
}
\end{figure}
%%%%%%%%%%%%%%%%%%%%%%%%%%%%%%%%%%%%%%%%%%%%%%%%%%%%%%%%%%%%%%%%%%%%%%%%%%  

\end{document}

